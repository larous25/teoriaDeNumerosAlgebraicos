\documentclass{article}
\usepackage[utf8]{inputenc}
\usepackage{amsmath}

\title{Semillero - Teoria de numeros álgebraicos}
\author{larous025 }
\date{October 2019}

\begin{document}

\maketitle

\section{}
Tomamos una recta $l$ que pasa por $p \land q$ \\
Como $y^2 = f(x)$ es una cubica la recta $l$ y la curva $E$ se tocan en tres $(p,q,\widetilde{r})$ 
\section{}
\begin{equation*}
\widetilde{r}=(a,b)     
\end{equation*}
\begin{equation*}
    r:= p \oplus q := (a,-b)
\end{equation*}
\begin{equation*}
    y^2=x^3-2x p=(0,0) q=(-1,-1) l: y=X
\end{equation*}
\begin{equation*}
    E \cap l = {p,q,\widetilde{r}}
\end{equation*}
\begin{equation*}
    E \cap l = {(0,0),(-1,-1),(2,2)}
\end{equation*}
\begin{equation*}
    p \oplus q=(2,-2)
\end{equation*}
\section{}
Cuando los dos puntos no son diferentes son iguales
$Q\oplus Q$ :recta tangente a la cubica 
\begin{equation*}
    y^{2}= x^{3}-2x
    Q=(-1,1)
    m=\frac{dy}{dx}(-1,-1)
    2y\frac{dy}{dx}=3x^{2}-2
    \frac{dy}{dx}= \frac{3x^{2}-2}{2y}
    es (-1,-1)
    \frac{dy}{dx} \left| (-1,-1) \right. = \frac{3-2}{-2}= \frac{-1}{2}
    \frac{y-(-1)}{x-(-1)} = \frac{-1}{2}
    y+1 = \frac{-1}{2}(x+1)
    y=\frac{-y}{2}-\frac{3}{2} = \frac{-(x+5)}{2}
\end{equation*}
\begin{equation*}
    l_{2}\cupE: \frac{x^{2}+6x+9}{4}=x^{3}-2x
    4x^{3}-8x-x^{2}-6x-9=0
    4(x+1)^2(x--\theta)
    \theta = \frac{9}{4}
    \widetilde{R} = (\frac{9}{4}, \frac{21}{8})
    Q\oplus Q = (\frac{9}{4}, \frac{21}{8})
\end{equation*}
\end{document}
