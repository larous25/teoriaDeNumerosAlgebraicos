\documentclass{article}
\usepackage[utf8]{inputenc}
\usepackage{amsmath}

\title{Semillero - Teoria de numeros álgebraicos}
\author{larous025 }
\date{October 2019}

\begin{document}

\maketitle

\section{}
Tomamos una recta $l$ que pasa por $p \land q$ \\
Como $y^2 = f(x)$ es una cubica la recta $l$ y la curva $E$ se tocan en tres $(p,q,\widetilde{r})$ 
\section{}
\begin{equation*}
\widetilde{r}=(a,b)     
\end{equation*}
\begin{equation*}
    r:= p \oplus q := (a,-b)
\end{equation*}
\begin{equation*}
    y^2=x^3-2x p=(0,0) q=(-1,-1) l: y=X
\end{equation*}
\begin{equation*}
    E \cap l = {p,q,\widetilde{r}}
\end{equation*}
\begin{equation*}
    E \cap l = {(0,0),(-1,-1),(2,2)}
\end{equation*}
\begin{equation*}
    p \oplus q=(2,-2)
\end{equation*}
\section{}

\end{document}
