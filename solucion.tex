\documentclass{article}
\usepackage[utf8]{inputenc}
\usepackage{amsmath}
\usepackage{geometry}
\usepackage{graphicx}


\title{Semillero - Teoria de numeros álgebraicos}
\author{- Brian E. Bustos B. \\
- Nicolas Camargo \\
- Miguel Gomez \\
- Luis Tarazona }
\date{Noviembre 2019}

\makeatletter         
\def\@maketitle{
\raggedright

\begin{center}
{\Huge \bfseries \@title }\\[4ex] 
{\Large  \@author}\\[4ex] 
\@date\\[8ex]
\includegraphics[width = 40mm]{logo.png}
\end{center}}
\makeatother

\begin{document}

\maketitle

\section{}
Tomamos una recta $l$ que pasa por $p \land q$ \\
Como $y^2 = f(x)$ es una cubica la recta $l$ y la curva $E$ se tocan en tres $(p,q,\widetilde{r})$ 
\section{}
\begin{equation*}
\widetilde{r}=(a,b)     
\end{equation*}
\begin{equation*}
    r:= p \oplus q := (a,-b)
\end{equation*}
\begin{equation*}
    y^2=x^3-2x p=(0,0) q=(-1,-1) l: y=X
\end{equation*}
\begin{equation*}
    E \cap l = {p,q,\widetilde{r}}
\end{equation*}
\begin{equation*}
    E \cap l = {(0,0),(-1,-1),(2,2)}
\end{equation*}
\begin{equation*}
    p \oplus q=(2,-2)
\end{equation*}
\section{}
Cuando los dos puntos no son diferentes son iguales
$Q\oplus Q$ :recta tangente a la cubica 
\begin{equation*}
    y^{2}= x^{3}-2x
\end{equation*}
\begin{equation*}
    Q=(-1,1)
\end{equation*}
\begin{equation*}
    m=\frac{dy}{dx}(-1,-1)
\end{equation*}
\begin{equation*}
    2y\frac{dy}{dx}=3x^{2}-2
\end{equation*}
\begin{equation*}    
    \frac{dy}{dx}= \frac{3x^{2}-2}{2y}
\end{equation*}
\begin{equation*}
    es (-1,-1)
\end{equation*}
\begin{equation*}
    \frac{dy}{dx} \left| (-1,-1) \right. = \frac{3-2}{-2}= \frac{-1}{2}
\end{equation*}
\begin{equation*}
    \frac{y-(-1)}{x-(-1)} = \frac{-1}{2}
\end{equation*}
\begin{equation*}
    y+1 = \frac{-1}{2}(x+1)
\end{equation*}
\begin{equation*}
    y=\frac{-y}{2}-\frac{3}{2} = \frac{-(x+5)}{2}
\end{equation*}
\begin{equation*}
    l_{2} \cup E: \frac{x^{2}+6x+9}{4} = x^{3}-2x
\end{equation*}
\begin{equation*}
    4x^{3}-8x-x^{2}-6x-9=0
\end{equation*}
\begin{equation*}
    4(x+1)^2(x--\theta)
\end{equation*}
\begin{equation*}
    \theta = \frac{9}{4}
\end{equation*}
\begin{equation*}
    \widetilde{R} = (\frac{9}{4}, \frac{21}{8})
\end{equation*}
\begin{equation*}
    Q\oplus Q = (\frac{9}{4}, \frac{21}{8})
\end{equation*}
\end{document}